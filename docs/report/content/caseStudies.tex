\chapter{Case studies}
\label{sec:caseStudies}

% present a balanced argument for why a decentralised chat i.e. pros/cons decentralised communication protocols
% present a balanced argument for why a decentralised wiki i.e. pros/cons decentralised encyclopedic information protocols

% Intro
While so far we have explored the design and implementation of the framework, it is important to also explore how the framework will be used to develop decentralised applications. Three example decentralised applications were implemented alongside Butter to test features and the development experience. One of the simplest examples was a reverse echo application where a user submits a string to the service, and it returns a reverse of the string, the processing happening elsewhere on the network. While this is not an interesting application, it provided grounds for testing. Two significantly more compelling case studies were explored: a decentralised chat application and wiki application. The chat and wiki application demonstrate the advantages of decentralisation for communication and high availability information storage respectively.

% Decentralised communication case study - simple peer-to-peer chat
Decentralised chat applications are a fascinating service case study as decentralised communication goes back to the original roots of the Internet's inception. ARPNET the predecessor to the Internet, which laid the foundational groundwork for the technology, was conceived as a ``computer communications system without a central core, with no headquarters or base of operations that could not be attacked and destroyed by enemies thus blacking out the entire network in one fell swoop"\cite{featherly2022arpnet}. To contextualise, this research was carried out at the height of the cold war. In addition, it seems intuitive to design communication systems in a decentralised way. As humans, this is typically how we navigate and communicate in our daily lives. However, as systems grow in user-base, it becomes natural to employ structured mechanisms to manage communication complexity.

In recent decades we have seen the rise of cloud services, i.e.\ services provided by third-parties\cite{redhat2022cloud}. Since this shift, we have blindly moved our habits to these services, for practicality as they are presented as highly reliable and dependable. Many of these services become viable businesses by either asking for user subscriptions in exchange for use or more commonly monetising their service through advertisement revenue. The resulting services, are often propriety, centralised and out of the user's control. This poses an interesting question: should we allow ourselves to become highly dependent on cloud services, specially for communication services that are so fundamental to our daily lives. If we can trust the cloud service providers, the problem is mute, if not, this model of service delivery cannot be sustained\cite{benkler2016degrees}.

Butter's demo chat application allows direct peer-to-peer communication between peers on the same LAN. This application is still rudimentary and will be further extended to enable direct peer-to-peer communication across the internet once the Wider discovery module is further tested.

% Decentralised information storage - simple wiki
The other case study for a Butter application is a wiki. A wiki can be thought of as a, typically community information driven, encyclopedia service. It is a particularly pertinent case study for an application of a fault-tolerant decentralised information service. The information in a wiki tends to be provided by contributing members of the community. There is no single author and hence it can be difficult to attribute information to any single figurehead. This model of information service brings about some interesting questions: Who owns the information? Who is responsible for the information and the consequences of its dissemination? Who maintains and hosts the service? In certain scenarios, the service can be of immense value to its users and society more broadly, so, it is important that the service be dependable and highly available. In addition, ubiquitous access to information may be desirable for a service, so it needs to be designed to be devoid of central control and censorship.

Some wikis are hosted and maintained by purely altruistic organisations and individuals. However, this is not always the case, and in some cases it is wise to consider whether we should trust private third-parties to have the interests of the community at heart. This problem could be avoided if the service were delivered by an autonomous decentralised network where, like the information, the infrastructure would be community contributed as well.

In Butter's wiki demo, the application behaves similarly to a typical wiki. Users can publish and retrieve informative articles, but it is significant to note that there is no central server, the information is not stored in a central index or hosted by a single third-party. Each node instance running the service has never explicitly been made aware of others, yet the service is still delivered. Nodes work together to maintain the information using the PCG mechanisms and the Information retrieval module handles decentralised search across the network removing the need for a central database. This has the effect of creating an autonomous service.

Having said that, there are still severe limitations with Butter and decentralised service delivery more broadly. Something that has been made clear throughout the literature is that the problem of scalable decentralised services has yet to be solved. In addition, it is important to note that not all services should or need to be autonomously delivered. There is a case that services involving personal information should not be autonomous.

% Blog
Take the example of a blog or social media sharing platform. These platforms may deal with personal information which should be considered with caution. Butter is not yet equipped to deal with personal information. Currently, there are no inbuilt encryption mechanisms and while the autonomy means information is devoid of malicious control it makes controlling personal information on the network difficult. This is further discussed in the legal, social and ethical considerations section (see~\ref{legalSocialEthical}) of the project.

% Probabilities
% Something to note is that in a complex environment, a service will always have a probability associated with its successful delivery. Decentralised systems can seem fault prone because they openly disclose that there operations have a certain probability of functioning. Cloud services tend to present themselves as a certainty, but like all systems they also have a probability of failure. Hence successufly retrieving a piece of information on Butter or Google drive both have probabilities associated with them.


