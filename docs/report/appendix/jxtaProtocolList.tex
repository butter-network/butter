\chapter{List of JXTA protocols}
\label{jxtaProtocols}

Project JXTA has defined six protocols so far.
\begin{itemize}
    \item Peer Discovery Protocol—enables a peer to find advertisements on
other peers and can be used to find any of the peer, peer group, or adver-
tisements.This protocol is the default discovery protocol for all peer groups,
including theWorld Peer Group. It is conceivable that someone might want
to develop a premium discovery mechanism that might or might not choose
to leverage this default protocol, but the inclusion of this default protocol
means that all JXTA peers can understand each other at the very basic level.
Peer discovery can be done with or without specifying a name for
either the peer to be located or the group to which peers belong.When
no name is specified, all advertisements are returned.
\item Peer Resolver Protocol—enables a peer to send and receive gener-
ic queries to search for peers, peer groups, pipes, and other informa-
tion.Typically, this protocol is implemented only by those peers that
have access to data repositories and offer advanced search capabilities.
\item Peer Information Protocol—allows a peer to learn about the capabil-
ities and status of other peers. For example, a ping message can be sent to
see if a peer is alive. A query can also be sent regarding a peer’s proper-
ties where each property has a name and a value string.
\item Peer Membership Protocol—allows a peer to obtain group member-
ship requirements, to apply for membership and receive a membership cre-
dential along with a full group advertisement, to update an existing mem-
bership or application credential, and to cancel a membership or an
application credential. Authenticators and security credentials are used to
provide the desired level of protection.
\item Pipe Binding Protocol—allows a peer to bind a pipe advertisement to
a pipe endpoint, thus indicating where messages actually go over the pipe.
In some sense, a pipe can be viewed as an abstract, named message queue
that supports a number of abstract operations such as create, open, close,
delete, send, and receive. Bind occurs during the open operation, whereas
unbind occurs during the close operation.
\item Endpoint Routing Protocol—allows a peer to ask a peer router for available
routes for sending a message to a destination peer. For example, when two
communicating peers are not directly connected to each other, such as when
they are not using the same network transport protocol or when they are sep-
arated by firewalls or NATs, peer routers respond to queries with available
route information—that is, a list of gateways along the route.Any peer can
decide to become a peer router by implementing the Peer Endpoint Protocol.
\end{itemize}